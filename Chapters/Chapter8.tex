% Chapter 8

\chapter{Conclusions and Further Work} % Main chapter title

\label{Chapter8} % For referencing the chapter elsewhere, use \ref{Chapter1} 

\lhead{Chapter 8. \emph{Conclusions and Further Work}} % This is for the header on each page - perhaps a shortened title

%----------------------------------------------------------------------------------------
This thesis presents a novel approach in information visualisation. Firstly,  at the beginning of this chapter, the research model has been summarised; secondly, the scope of the system is discussed and lastly, the recommendations for future work are described.

The huge amount of data on the internet and in small and medium sized businesses makes it difficult to manipulate or re-use the collected information in a cost effective manner. How could a large amount of information be gathered into one place and be understood quickly and effectively? The major contributor to this problem is the computer hardware and software, as the capacity and storage of data has tremendously improved, the processing power outstandingly increased and it is on the rise day by day. The problem is not with generating or creating data, but it is definitely with the quality of the data. The definition of quality data is a very broad topic as the requirements or understanding of data varies from user to user or system to system. Therefore, the problem is not why such a large amount of data is created or generated and how one can restrict it, but it is definitely with how effectively this large amount of data could be manipulated for various outputs, to understand what needs to be understood and to explore what needs to be discovered.

The solution to data problems revolve around intelligent tools and applications. Converting raw or refined data into a format where many things could be understood with a minimum of effort and the famous phrase: a picture is worth a thousand words. The complex data manipulation and problem of understanding lie with information visualisation. A good visualisation system will cover important aspects such as examining full data sets to explore data fully; the ability to visualise non-aggregated elements and additional information available to a user upon request. The contributions section will discuss how the visualisation model and the proposed system tackled data problems highlighted in this research. 

\section{Contributions}

This research has contributed to various aspects of data manipulation problems and information visualisation using web mashup technologies. All of the research objectives are achieved and validated through three experiments presented in this research. The contributions of this study are highlighted below:

\begin{itemize}

\item \textbf{Information Visualisation Model:}\\

The contribution of this research revolves around the four layers of the visualisation model. The first layer, acquisition and data analysis, demonstrates how complex data could be refined, filtered and mined for data representation purposes. The system models and flow charts demonstrated how data could be retrieved by the users in an enterprise environment and passed on to the data representation layer. The solution to data sets which were not normalised by Enterprise 2.0 environments are processed by a tool called Visualixer, which intelligently converts and validates, refines and filters data into a normalised form and which is then made available to the data representation layer for visual representation. \\

There are various important contributions achieved in the data representation layer:  (i) the introduction of an existing mash-up system for data representation; and (ii) the variety of data expression which helped in exploring data from various angles.\\

\item \textbf{Multi-attribute Information Visualisation:} \\
The multi-attribute data visualisation was achieved through a data mash-up system with resource friendly technologies. The end users were given extensive options to explore data in detail for in-depth analysis. The multi-attribute visualisation techniques help users to relate data with other elements in the data sets. For instance, sales staff performance is measured against product sales. This comparison helps explore which staff member is best at selling which products. This gives a good understanding to the management for effective resource utilisation. The users are empowered to inspect data with multi-attribute investigation leading to the exploration of hidden facts and trends.\\

\item \textbf{Multi-Dimension Information  Visualisation}:\\

Multi-dimension data visualisation gives users a different perspective to analyse and explore a large amount of complex data, which was not possible with simple one-dimensional data visualisation. The data elements are analysed visually with multi-dimension information representation. The information visualisation model has delivered multi-dimension information visualisation enabling decision makers or data analysers better equipped about data related problems or questions. \\

\item \textbf{Transactional Tagging Information Visualisation:}\\ 

Transactional tagging is a very fresh and novel contribution in information visualisation, particularly in Enterprise 2.0 environments. Transactional tagging visualisation, where each data element is tagged by the data analysis model and then visualised at the representation layer, assisting users to distinguish visualised data elements through tagging. Various elements are tagged which are then visualised at the representation layer showing vital relation attributes, for example, visualising staff members' sales of various sports tours. The analyser can quickly identify what tags have an influence on staff or which staff have an influence on tags. This shows which tags are popular in certain areas (cities and towns) in the country, where the business could focus, and take it to the next level.

\item \textbf{Multi-Coordinate Information Visualisation:} \\

Multi-coordinate visualisation is another unique aspect of the research. The purpose of this type of visualisation is to enable users to see the same retrieved information from the data set or database but with different visualisation styles at the same time. For instance, the output could be shown in a pie chart, and the same information next to the graph could be visualised in an area chart. The user can analyse data from a picture which is easy on the eye or more meaningfully for the data elements. This technique has been used as a fresh approach on transactional data sets facilitating small and medium sized businesses when exploring complex data sets.\\


\item \textbf{Linked Data Visualisation:}\\

Linked data visualisation is an exclusive and novel contribution in this study. In the database, or any other form other than the visualised form of data, linking elements of data should not be a big problem as we often normalise relational databases in relating one table to another so that the database or data elements make more sense. The linked data visualisation feature highlighted the need for such tools to understand data more precisely for more trends and analysis exploration. Linked data visualisation works with multi-coordinate and multi-attribute visualisation, where similar or related data elements visualised and presented, highlighting the relationships between different visualised aspects. The linking of data without visualisation could be done with various approaches but linking relation elements of data in visualised form is a fresh and novel approach. The idea is validated through various examples in Chapter 6 and Chapter 7. 

\item \textbf{User Interactivity:} \\

The interactive interfaces are introduced at the interact++ layer with the help of modern HTML5 and CSS3 powered by AJAX and JQuery applications. These tools are designed to make user interaction a lot more interesting and easier for the user. The web applications are not considered to be more user friendly when compared to desktop applications, where better user interaction has been a feature for quite a long time; however the interfaces introduced and developed for various experiments in this research are one of the strongest aspects because in the broad visualisation model where complex visualisation solutions were outcomes of advance research,  simple visualisation libraries were included in applications. There was a greater need for a third dimension application where complex data visualisation problems are addressed at the end user’s fingertips. The Visualixer tool is a tangible outcome of this research.

\item \textbf{Results Repositories:} \\

Most of the existing visualisation tools do not really focus on the re-use of the visualised information. The export feature of visualised information has been brought back to the discussion and its importance for comparison or version control purposes are highlighted. The history layer of the visualisation model plays a vital role in complex data comparison in a visaulised form.

\end{itemize}

\section{Further Work}

Further work and enhancements in the system are mostly related to acquisition, the data analysis layer and the data representation layer, which are also the strongest aspects of this research. The ability to process unstructured complex data for visualisation is a limitation as the system currently processes semi-structural data in a non-enterprise environment (the environment where data structure is not in a known or normalised form). However, the proposed system serves its purpose in enterprise environments with the scope of improvements in data selection and versatility. The interact++ layer with the ability to give more freedom to the end user in data selection for visualisation is another area which requires further enhancement. The processing functions of the data analysis model needs to be updated for larger data set manipulation. The representation layer could be further improved through more interactive graphs being added to the library. Multi-attribute visualisation is currently not supported by all graphs in the system library, which is not a major flaw but requires further adjustment. The linked data visualisation requires more development in order to achieve precise visualised outcomes of the relational data elements. The linked data visualisation only works with multi-coordinate graphs and charts; however, this could be addressed and a solution to this problem has been suggested in the future work and development paragraph below.

The user data element selection requires a robust model where users could select various data elements in the data representation model. The current system stores preferred information in the database for future re-use in an enterprise environment, but would introduce an efficient model which would focus on data elements selected by the user before visualisation process. This approach would make the system more dynamic and a user selection oriented tool. The more effectively the message is conveyed, the better the understanding that could be derived from the data. The data representation layer of the visualisation model requires further development as some data sets visualisation in all charts might not be possible which is why enhancing visualisation abilities of the model would make a big difference. The multi-attribute visualisation could be a data problem in its own right, as the more varied the information is when transferred into visualised form, the better are the chances of understanding and exploring data trends. 

The linked data visualisation, displaying comparative data within the same graph, would make the process more interactive to the end user, and more data elements of relational type would be available on-demand to the user. The colour and visual form in the linked data chart would also be valuable in generating quality graphs with relation to the data elements. The on-demand information analysis and visualisation in a non-enterprise environment for non-relational data is usually not an easy process to represent in visual form, which could be considered as future work with potential to solve complex data problems. The mobile application for the non-enterprise visualiser will give freedom to mobile users. The non-enterprise visualiser is a cloud based system, but added responsive features will add more value and will target a broader audience across different platforms. 

Finally, the information visualisation model proposed and validated in this research is a complete set of facilities for data related problems and questions. The technique will assist in analysing any type and size of data, helping small and medium sized businesses and organisations to explore data further to show trends and progressions. The Visualixer tool will enable individuals to process data with ease and with options to export for external usage.
